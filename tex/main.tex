\documentclass[
  12pt,
  openright,
  twoside,
  a4paper,
  english,
  french,
  spanish,
  brazil
]{abntex2}

\usepackage{lmodern}
\usepackage[T1]{fontenc}
\usepackage[utf8]{inputenc}
\usepackage{indentfirst}
\usepackage{color}
\usepackage{graphicx}
\usepackage{microtype}
\usepackage[brazilian,hyperpageref]{backref}
\usepackage{abntex2cite}
\usepackage{customization}

\renewcommand{\backrefpagesname}{Citado na(s) página(s):~}
\renewcommand{\backref}{}
\renewcommand*{\backrefalt}[4]{
	\ifcase #1
		Nenhuma citação no texto.
	\or
		Citado na página #2.
	\else
		Citado #1 vezes nas páginas #2.
	\fi
}

\titulo{Guia de Acessibilidade Digital}
\autor{Grupo 06}
\local{Brasília, Brasil}
\data{2024, v0.2.0}
\orientador{Rejane Figueiredo}
\instituicao{
  Universidade de Brasília
  \par
  Faculdade do Gama
}
\tipotrabalho{Relatório técnico}
\preambulo{
  Trabalho submetido à disciplina de Interação Humano-Computador da Universidade
  de Brasília, ministrada pela professora Rejane Figueiredo.
}

\definecolor{blue}{RGB}{41,5,195}

\makeatletter
\hypersetup{
  pdftitle={\@title},
  pdfauthor={\@author},
  pdfsubject={\imprimirpreambulo},
  pdfcreator={LaTeX with abnTeX2},
  pdfkeywords={abnt}{latex}{abntex}{abntex2}{trabalho acadêmico},
  colorlinks=true,
  linkcolor=blue,
  citecolor=blue,
  filecolor=magenta,
  urlcolor=blue,
  bookmarksdepth=4
}
\makeatother

\makeatletter
\setlength{\@fptop}{5pt}
\makeatother

\newcommand{\quadroname}{Quadro}
\newcommand{\listofquadrosname}{Lista de quadros}

\newfloat[chapter]{quadro}{loq}{\quadroname}
\newlistof{listofquadros}{loq}{\listofquadrosname}
\newlistentry{quadro}{loq}{0}

\setfloatadjustment{quadro}{\centering}
\counterwithout{quadro}{chapter}
\renewcommand{\cftquadroname}{\quadroname\space}
\renewcommand*{\cftquadroaftersnum}{\hfill--\hfill}

\setfloatlocations{quadro}{hbtp}

\setlength{\parindent}{1.3cm}
\setlength{\parskip}{0.2cm}

\makeindex

\begin{document}

\selectlanguage{brazil}
\frenchspacing

\pretextual
\imprimircapa
\imprimirfolhaderosto*

\begin{agradecimentos}
  Os agradecimentos principais são direcionados à professora Rejane Figueiredo,
  que ministrou a disciplina de Interação Humano-Computador. Sua dedicação e
  paciência foram fundamentais para o desenvolvimento deste guia.
\end{agradecimentos}

% ---
% RESUMOS
% ---

% resumo em português
\setlength{\absparsep}{18pt} % ajusta o espaçamento dos parágrafos do resumo
\begin{resumo}
 Segundo a \citeonline[3.1-3.2]{NBR6028:2003}, o resumo deve ressaltar o
 objetivo, o método, os resultados e as conclusões do documento. A ordem e a extensão
 destes itens dependem do tipo de resumo (informativo ou indicativo) e do
 tratamento que cada item recebe no documento original. O resumo deve ser
 precedido da referência do documento, com exceção do resumo inserido no
 próprio documento. (\ldots) As palavras-chave devem figurar logo abaixo do
 resumo, antecedidas da expressão Palavras-chave:, separadas entre si por
 ponto e finalizadas também por ponto.

  \textbf{Palavras-chave}:
  Interação Humano-Computador. Guia de acessibilidade digital. Design de
  Serviços. Pessoas com deficiência. Tecnologia da informação. Experiência do
  usuário. Acessibilidade.
\end{resumo}

\cleardoublepage

\pdfbookmark[0]{\listofquadrosname}{loq}
\listofquadros*
\cleardoublepage

\pdfbookmark[0]{\listtablename}{lot}
\listoftables*
\cleardoublepage

\begin{siglas}
  \item[IHC] Interação Humano-Computador
  \item[TI] Tecnologia da Informação
  \item[PCD] Pessoas com Deficiência
  \item[IBGE] Instituto Brasileiro de Geografia e Estatística
  \item[WCAG] \textit{Web Content Accessibility Guidelines}
  \item[W3C] \textit{World Wide Web Consortium}
  % \item[DS] Design de Serviços
  % \item[eMAG] Modelo de Acessibilidade em Governo Eletrônico
  % \item[ASES] Avaliador e Simulador de Acessibilidade em Sítios
  % \item[UCD] \textit{User-Centered Design}
  % \item[UX] \textit{User Experience}
  % \item[ISO] \textit{International Organization for Standardization}
  % \item[WAI] \textit{Web Accessibility Initiative}
  % \item[WAI-ARIA] \textit{Web Accessibility Initiative - Accessible Rich Internet Applications}
  % \item[IHIP] \textit{Intangible, Heterogeneous, Inseparable and Perishable}
  % \item[DOM] \textit{Document Object Model}
  % \item[HTML] \textit{HyperText Markup Language}
  % \item[CSS] \textit{Cascading Style Sheets}
  % \item[XML] \textit{eXtensible Markup Language}
  % \item[XHTML] \textit{eXtensible HyperText Markup Language}
\end{siglas}

\begin{simbolos}
  \item[$ \Gamma $] Letra grega Gama
  \item[$ \Lambda $] Lambda
  \item[$ \zeta $] Letra grega minúscula zeta
  \item[$ \in $] Pertence
\end{simbolos}

\pdfbookmark[0]{\contentsname}{toc}
\tableofcontents*
\cleardoublepage

\textual

\chapter{Introdução}

Este guia de acessibilidade digital foi criado como resultado da disciplina de
IHC da Faculdade do Gama da Universidade de Brasília, com o intuito de aplicar e
demonstrar os conhecimentos dos estudantes do grupo sobre os conteúdos
apresentados na disciplina durante o primeiro semestre de 2024.

O objetivo é oferecer a todo profissional da área de TI subsídios teóricos e
práticos, tanto no âmbito documental e ferramental, para que o desenvolvedor
possa compreender os conceitos sobre IHC e aplique estratégias de transformação
digital acessível, considerando que o sistema a ser desenvolvido será adequado à
PCDs e implemente a estrutura de acessibilidade definida pelo Governo Federal.

Esse material se faz de grande importância para manter os conhecimentos
``frescos'' sobre acessibilidade na web, seja para o profissional que não
conhece o tema ter a oportunidade de estudá-lo, como para o mais experiente
consultar informações técnicas.

A acessibilidade digital importa pois, de acordo com números fornecidos pelo
IBGE, mais de 45 milhões de cidadãos brasileiros possuem alguma deficiência
\cite{DAP:Guia-de-Boas-Praticas}. Dificuldades para interagir e compreender o
conteúdo de uma aplicação web, não conseguir concluir cadastros ou pagamentos e
falta de botões acessíveis podem se tornar um grande empecilho ao indivíduo.

Quando conteúdos presentes na internet, em aplicativos para telefone, vídeos nas
plataformas digitais e conteúdos televisivos são acessíveis, todas as pessoas
são beneficiadas, especialmente as que possuem algum tipo de deficiência,
oferecendo conforto e segurança a este público-alvo.

As diretrizes do WCAG do W3C oferecem um caminho acessível à web, com a
acessibilidade sendo tratada em três grandes pontos: design, conteúdo e
desenvolvimento. O conceito de \textit{device-agnostic} é o qual o WCAG se
baseia, o que significa que a s recomendações se referem, com contrastes e
botões sendo acessíveis tanto no computador, como nos aparelhos móveis.

Ao final desta etapa, espera-se do profissional de TI que:

\begin{itemize}
  \item conheça a necessidade de ser acessível na contemporaneidade;
  \item identifique a importância do material para consultas do time de TI; e
  \item
    compreenda o impacto que introduzir acessibilidade no cotidiano do
    desenvolvimento pode acarretar na população.
\end{itemize}

\chapter{Conclusão}

\postextual

\bibliography{refs}

%---------------------------------------------------------------------
% INDICE REMISSIVO
%---------------------------------------------------------------------
\phantompart
\printindex
%---------------------------------------------------------------------

\end{document}
