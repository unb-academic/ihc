\documentclass[
  12pt,
  openright,
  twoside,
  a4paper,
  english,
  french,
  spanish,
  brazil
]{abntex2}

\usepackage{lmodern}
\usepackage[T1]{fontenc}
\usepackage[utf8]{inputenc}
\usepackage{indentfirst}
\usepackage{color}
\usepackage{graphicx}
\usepackage{microtype}
\usepackage[brazilian,hyperpageref]{backref}
\usepackage{abntex2cite}
\usepackage{lipsum}
\usepackage{customization}

\renewcommand{\backrefpagesname}{Citado na(s) página(s):~}
\renewcommand{\backref}{}
\renewcommand*{\backrefalt}[4]{
	\ifcase #1
		Nenhuma citação no texto.
	\or
		Citado na página #2.
	\else
		Citado #1 vezes nas páginas #2.
	\fi
}

\titulo{Guia de Acessibilidade Digital}
\autor{Grupo 06}
\local{Brasília, Brasil}
\data{2024, v0.2.0}
\orientador{Rejane Figueiredo}
\instituicao{
  Universidade de Brasília
  \par
  Faculdade do Gama
}
\tipotrabalho{Relatório técnico}
\preambulo{
  Trabalho submetido à disciplina de Interação Humano-Computador da Universidade
  de Brasília, ministrada pela professora Rejane Figueiredo.
}

\definecolor{blue}{RGB}{41,5,195}

\makeatletter
\hypersetup{
  pdftitle={\@title},
  pdfauthor={\@author},
  pdfsubject={\imprimirpreambulo},
  pdfcreator={LaTeX with abnTeX2},
  pdfkeywords={abnt}{latex}{abntex}{abntex2}{trabalho acadêmico},
  colorlinks=true,
  linkcolor=blue,
  citecolor=blue,
  filecolor=magenta,
  urlcolor=blue,
  bookmarksdepth=4
}
\makeatother

\makeatletter
\setlength{\@fptop}{5pt}
\makeatother

\newcommand{\quadroname}{Quadro}
\newcommand{\listofquadrosname}{Lista de quadros}

\newfloat[chapter]{quadro}{loq}{\quadroname}
\newlistof{listofquadros}{loq}{\listofquadrosname}
\newlistentry{quadro}{loq}{0}

\setfloatadjustment{quadro}{\centering}
\counterwithout{quadro}{chapter}
\renewcommand{\cftquadroname}{\quadroname\space}
\renewcommand*{\cftquadroaftersnum}{\hfill--\hfill}

\setfloatlocations{quadro}{hbtp}

\setlength{\parindent}{1.3cm}
\setlength{\parskip}{0.2cm}

\makeindex

\begin{document}

\selectlanguage{brazil}
\frenchspacing

\pretextual
\imprimircapa
\imprimirfolhaderosto*

\begin{agradecimentos}
  Os agradecimentos principais são direcionados à professora Rejane Figueiredo,
  que ministrou a disciplina de Interação Humano-Computador. Sua dedicação e
  paciência foram fundamentais para o desenvolvimento deste guia.
\end{agradecimentos}

% ---
% RESUMOS
% ---

% resumo em português
\setlength{\absparsep}{18pt} % ajusta o espaçamento dos parágrafos do resumo
\begin{resumo}
 Segundo a \citeonline[3.1-3.2]{NBR6028:2003}, o resumo deve ressaltar o
 objetivo, o método, os resultados e as conclusões do documento. A ordem e a extensão
 destes itens dependem do tipo de resumo (informativo ou indicativo) e do
 tratamento que cada item recebe no documento original. O resumo deve ser
 precedido da referência do documento, com exceção do resumo inserido no
 próprio documento. (\ldots) As palavras-chave devem figurar logo abaixo do
 resumo, antecedidas da expressão Palavras-chave:, separadas entre si por
 ponto e finalizadas também por ponto.

 \textbf{Palavras-chave}: latex. abntex. editoração de texto.
\end{resumo}

\cleardoublepage

\pdfbookmark[0]{\listofquadrosname}{loq}
\listofquadros*
\cleardoublepage

\pdfbookmark[0]{\listtablename}{lot}
\listoftables*
\cleardoublepage

\begin{siglas}
  \item[IHC] Interação Humano-Computador
  \item[IBGE] Instituto Brasileiro de Geografia e Estatística
  \item[DS] Design de Serviços
  \item[TI] Tecnologia da Informação
  \item[PCD] Pessoas com Deficiência
  \item[eMAG] Modelo de Acessibilidade em Governo Eletrônico
  \item[ASES] Avaliador e Simulador de Acessibilidade em Sítios
  \item[UCD] \textit{User-Centered Design}
  \item[UX] \textit{User Experience}
  \item[ISO] \textit{International Organization for Standardization}
  \item[WCAG] \textit{Web Content Accessibility Guidelines}
  \item[W3C] \textit{World Wide Web Consortium}
  \item[WAI] \textit{Web Accessibility Initiative}
  \item[WAI-ARIA] \textit{Web Accessibility Initiative - Accessible Rich Internet Applications}
  \item[IHIP] \textit{Intangible, Heterogeneous, Inseparable and Perishable}
  \item[DOM] \textit{Document Object Model}
  \item[HTML] \textit{HyperText Markup Language}
  \item[CSS] \textit{Cascading Style Sheets}
  \item[XML] \textit{eXtensible Markup Language}
  \item[XHTML] \textit{eXtensible HyperText Markup Language}
\end{siglas}

\begin{simbolos}
  \item[$ \Gamma $] Letra grega Gama
  \item[$ \Lambda $] Lambda
  \item[$ \zeta $] Letra grega minúscula zeta
  \item[$ \in $] Pertence
\end{simbolos}

\pdfbookmark[0]{\contentsname}{toc}
\tableofcontents*
\cleardoublepage

\textual

\chapter{Introdução}

Este guia de acessibilidade digital foi criado como resultado da disciplina de
IHC da Faculdade do Gama da Universidade de Brasília, com o intuito de aplicar e
demonstrar os conhecimentos dos estudantes do grupo sobre os conteúdos
apresentados na disciplina durante o primeiro semestre de 2024.

O objetivo é oferecer a todo profissional da área de TI subsídios teóricos e
práticos, tanto no âmbito documental e ferramental, para que o desenvolvedor
possa compreender os conceitos sobre IHC e aplique estratégias de transformação
digital acessível, considerando que o sistema a ser desenvolvido será adequado à
pessoas com deficiência e implemente a estrutura de acessibilidade definida pelo
Governo Federal.

Esse material se faz de grande importância para manter os conhecimentos
``frescos'' sobre acessibilidade na web, seja para o profissional que não
conhece o tema ter a oportunidade de estudá-lo, como para o mais experiente
consultar informações técnicas.

A acessibilidade digital importa pois, de acordo com números fornecidos pelo
IBGE, mais de 45 milhões de cidadãos brasileiros possuem alguma deficiência
\cite{DAP:Guia-de-Boas-Praticas}. Dificuldades para interagir e compreender o
conteúdo de uma aplicação web, não conseguir concluir cadastros ou pagamentos e
falta de botões acessíveis podem se tornar um grande empecilho ao indivíduo.

Quando conteúdos presentes na internet, em aplicativos para telefone, vídeos nas
plataformas digitais e conteúdos televisivos são acessíveis, todas as pessoas
são beneficiadas, especialmente as que possuem algum tipo de deficiência,
oferecendo conforto e segurança a este público alvo.

As diretrizes do WCAG do W3C oferecem um caminho acessível à web, com a
acessibilidade sendo tratada em três grandes pontos: design, conteúdo e
desenvolvimento. O conceito de \textit{device-agnostic} é o qual o WCAG se
baseia, o que significa que a s recomendações se referem, com contrastes e
botões sendo acessíveis tanto no computador, como nos aparelhos móveis.

Ao final desta etapa, espera-se do profissional de TI que:

\begin{itemize}
  \item conheça a necessidade de ser acessível na contemporaneidade;
  \item identifique a importância do material para consultas do time de TI; e
  \item
    compreenda o impacto que introduzir acessibilidade no cotidiano do
    desenvolvimento pode acarretar na população.
\end{itemize}

\chapter{Como IHC interpreta DS}

O cenário do design de tecnologia da informação mudou significativamente, com
novas abordagens para gerenciar de forma eficaz sistemas de informação e
múltiplos \textit{stakeholders} surgindo a todo momento. Devido a isso, o design
de serviços já é uma realidade no mundo corporativo, consistindo em uma
metodologia para ajustar os pontos de contato com seus clientes, de forma que a
experiência deles possa ser enriquecida e processos internos da empresa sejam
otimizados paralelamente.

Historicamente, o design de serviços surge como uma abordagem crítica que evolui
na década de 1980 dentro dos estudos de gestão e operações, caracterizado por
focar em serviços intangíveis, heterogêneos e perecíveis, muitas vezes baseados
no IHIP, um modelo utilizado que caracteriza serviços em contraste com produtos
físicos. Nos anos 1990, a popularidade do design de serviços nas comunidades de
design enfatizou abordagens criativas e centradas no ser humano. Por fim, no
início dos anos 2000, a lógica dominante do serviço confundiu ainda mais a
distinção entre produtos tangíveis e serviços intangíveis, com foco na
co-criação de valor \cite{DAP:Guia-de-Boas-Praticas}.

\chapter{Histórico UCD a UX e ISOs}

A jornada do UCD, do UX e dos ISOs se entrelaça em uma história de evolução
constante, marcada pela busca incessante por criar experiências digitais
acessíveis e inclusivas para todos os usuários.

No início da década de 1980, a ergonomia dominava o campo do design de
interfaces, com foco na otimização da eficiência e da produtividade. No entanto,
o UCD surgiu como um paradigma emergente, defendendo a centralidade do usuário
no processo de design. Essa mudança de foco reconheceu a importância de entender
as necessidades, expectativas e comportamentos dos usuários para criar
interfaces intuitivas e satisfatórias
\cite{Adriana:A-Closer-Look-on-the-User-Centred-Design}.

Com o avanço da tecnologia e a diversificação das plataformas digitais, o UX se
consolidou como um conceito abrangente que engloba todos os aspectos da
experiência do usuário em um produto ou serviço. Essa abordagem holística
considera fatores como a usabilidade, a estética, a emoção e o significado,
buscando criar experiências memoráveis e positivas para os usuários.

O ISO, organização internacional dedicada à padronização, desempenha um papel
crucial na promoção da acessibilidade digital. Seus padrões, como o ISO
9241-11:2018, estabelecem diretrizes para o design e desenvolvimento de produtos
e serviços acessíveis a pessoas com deficiência \cite{ISO:9241-11:2018}. A
adoção desses padrões garante que as soluções digitais sejam inclusivas e
atendam às necessidades de todos os usuários.

O UCD, o UX e os ISOs se complementam e se reforçam em sua missão de criar
experiências digitais acessíveis. O UCD fornece a metodologia para entender as
necessidades dos usuários, o UX oferece a perspectiva holística da experiência
do usuário e os ISOs definem os padrões que garantem a implementação da
acessibilidade.

\chapter{Modelo de Acessibilidade}

O eMAG consiste em um conjunto de recomendações a ser considerado para que o
processo de acessibilidade dos sites e portais do governo brasileiro seja
conduzido de forma padronizada e de fácil implementação. Ele é coerente com as
necessidades brasileiras e em conformidade com os padrões internacionais, sendo
formulado para orientar profissionais que tenham contato com publicação de
informações ou serviços na internet a desenvolver, alterar  e/ou adequar
páginas, sites e portais, tornando-os acessíveis ao maior número de pessoas
possível.

\section{O Acesso de PCDs}

Quatro principais situações vivenciadas por usuários com deficiência são:

\begin{enumerate}
  \item
    acesso ao computador sem mouse (pessoas com deficiência visual, dificuldade
    de controle dos movimentos, paralisia ou amputação de um membro superior);
  \item
    acesso ao computador sem teclado (pessoas com amputações, grandes limitações
    de movimentos ou falta de força nos membros superiores);
  \item acesso ao computador sem monitor (pessoas com deficiência visual); e
  \item acesso ao computador sem áudio (pessoas com deficiência auditiva).
\end{enumerate}

Muitas vezes, a deficiência não é severa o suficiente a ponto de se tornar uma
barreira à utilização dos aparelhos tecnológicos. Entretanto, nos quatro tipos
de situações citadas anteriormente, barreiras de acessibilidade são encontradas,
dificultando o acesso ao conteúdo. Por serem as quatro principais situações
encontradas, muitos sistemas web englobam apenas políticas de desenvolvimento
que entreguem acessibilidade a tal público. Entretanto, por não serem apenas as
únicas deficiências que sofrem com acessibilidade, os projetos presentes na
internet devem englobar desde diferentes níveis de acessibilidade, faixa etária
e pouca experiência de computador, bem como ser compatível com as tecnologias
usadas para o acesso àquela informação específica.

É importante ressaltar que, por mais que também existam recursos responsáveis
por auxiliar PCDs na web, como os teclados adaptados, ampliadores de tela e
mouses especiais, tais ferramentas não são capazes de conferir toda a
acessibilidade que o usuário precisa para que ele tenha uma boa experiência.
Dessa forma, desenvolver a página de acordo com os padrões web e as
recomendações de acessibilidade são intrinsecamente importantes para todo
projeto.

\section{O Processo para Desenvolver Sistemas de Informação Acessíveis}

A acessibilidade na internet refere-se a garantir acesso facilitado para
qualquer indivíduo que, independente de suas condições mundanas, consiga
concluir seu objetivo. Dito isso, para desenvolver tais sistemas, três passos
são considerados:

\begin{enumerate}
  \item seguir os padrões da web;
  \item seguir as diretrizes ou recomendações de acessibilidade; e
  \item realizar a avaliação de acessibilidade.
\end{enumerate}

\section{Padrões Web}

Para que a acessibilidade necessária seja conferida ao sistema web, o código
deve estar nos padrões internacionais definidos pelo W3C. Estes padrões,
conhecidos como \textit{web standards}, são um conjunto de recomendações que
visam padronizar todo conteúdo que esteja na internet, possibilitando melhores
práticas de desenvolvimento de páginas. Normas HTML, XML, XHTML e CSS devem ser
seguidas tanto na formatação semântica, como na sintática, com cada elemento
utilizado tenho um significado apropriado, valor e propósito.

Esta conformidade permite que qualquer acesso à informação seja interpretado da
mesma forma e adequadamente por todos os indivíduos que a encontrem, seja por
meio dos navegadores, leitores de tela, dispositivos móveis, etc. Páginas que
não possuem código de acordo com os padrões do W3C apresentam comportamento
imprevisível e, na maioria das vezes, dificultam o acesso à informação.

\section{Recomendações de Acessibilidade}

As recomendações de acessibilidade explicam como tornar o conteúdo web acessível
a todas as pessoas, sendo importante para criadores de conteúdo e programadores
de ferramentas para criação de conteúdo, com sua principal documentação sendo a
WCAG, desenvolvida pelo consórcio W3C a partir da criação do WAI. O WAI ainda
desenvolveu especificações para aplicações web, ainda boa parte em status de
``rascunho'', chamados WAI-ARIA, que busca resolver muitos dos problemas da
camada de comportamento (DOM), sendo parte já implementada por alguns
navegadores. Por fim, o eMAG é o documento que norteia o desenvolvimento de
sites e portais acessíveis no âmbito do Governo Federal.

Apesar de tornarem a avaliação de acessibilidade mais rápida e menos trabalhosa,
os validadores automáticos por si só não determinam se um site está ou não
acessível, o que torna uma validação manual posterior necessária. Por meio da
validação manual, um checklist de validação humana é utilizado com o intuito de
que todos os problemas de acessibilidade em um site que não são detectados
mecanicamente pelos validadores sejam alcançados e tratados de forma correta,
com correções e novos testes após cada teste.

Os passos sugeridos para a avaliação de acessibilidade em um sistema de
informação web, de acordo com o eMAG, estão ligadas a seguir.

\begin{enumerate}
  \item Validar os \textbf{códigos do conteúdo HTML e das folhas de estilo}.
  \item
    Verificar o \textbf{fluxo de leitura da página}. A forma mais simples é
    inibir o CSS, imagens e scripts, lendo apenas o HTML da página. Boa
    parte dos navegadores possuem ferramentas ou extensões que permitem essa
    visualização.
  \item
    Realizar a validação automática de acessibilidade utilizando o ASES e outros
    avaliadores automáticos.
  \item
    Realizar a validação manual. A validação manual é uma etapa essencial na
    avaliação de acessibilidade de um site, já que os validadores automáticos
    não são capazes de detectar todos os problemas de acessibilidade em um site,
    pois muitos aspectos requerem um julgamento humano. Por exemplo, validadores
    automáticos conseguem detectar se o atributo para descrever imagens foi
    utilizado em todas as imagens do sites, mas somente uma pessoa poderá
    verificar se a descrição da imagem está adequada ao seu conteúdo. Para
    realizar uma validação manual efetiva, o desenvolvedor deverá ter
    conhecimento sobre as diferentes tecnologias, as barreiras de acessibilidade
    enfrentadas por pessoas com deficiência e as técnicas ou recomendações de
    acessibilidade. A validação manual deve ser feita preferencialmente com
    dispositivos de tecnologia assistiva como leitores de tela. Deve-se
    percorrer toda página apenas utilizando teclado, verificando comportamentos,
    atalhos, folhas alternativas de contraste, se os textos alternativos estão
    descritos de acordo com a imagem e seu contexto, entre outros.
  \item
    \textbf{Teste com usuários reais}. Outra etapa essencial da validação de uma
    página é a realização de testes com usuários reais (PCD ou limitações
    técnicas). Um usuário real poderá dizer se um sítio está realmente
    acessível, compreensível e com boa usabilidade e não simplesmente
    tecnicamente acessível. Quanto maior e mais diversificado o número de
    usuários reais participando da avaliação de acessibilidade, mais eficaz e
    robusto será o resultado.
\end{enumerate}

\section{Manutenção da Acessibilidade}

A promoção da acessibilidade é um processo contínuo, recomenda-se que testes
sejam realizados, de forma pontual, a cada alteração de conteúdo e validações
globais em espaços determinados de tempo. O intervalo depende de diversos
fatores, mas a recomendação é que se valide o site todo quando for feita a
atualização do Sistema de Gestão de Conteúdo ou mudança de desenho.

\section{Recursos de Acessibilidade}

Os padrões de acessibilidade compreendem recomendações ou diretrizes que visam
tornar o conteúdo web acessível a todas as pessoas, inclusive às PCD,
destinando-se aos autores de páginas, projetistas de sites e aos desenvolvedores
de ferramentas para criação de conteúdo. Para facilitar a implementação das
recomendações, no eMAG elas são separadas por seções de acordo com as
necessidades de implementação:

\begin{itemize}
  \item marcação;
  \item comportamento (DOM);
  \item conteúdo/informação;
  \item apresentação/design;
  \item multimídia; e
  \item formulário.
\end{itemize}


% ---
% Conclusão
% ---
\chapter{Conclusão}
% ---

\lipsum[31-33]

% ----------------------------------------------------------
% ELEMENTOS PÓS-TEXTUAIS
% ----------------------------------------------------------
\postextual
% ----------------------------------------------------------

\bibliography{refs}

% ----------------------------------------------------------
% Glossário
% ----------------------------------------------------------
%
% Consulte o manual da classe abntex2 para orientações sobre o glossário.
%
%\glossary

% ----------------------------------------------------------
% Apêndices
% ----------------------------------------------------------

% ---
% Inicia os apêndices
% ---
\begin{apendicesenv}

% Imprime uma página indicando o início dos apêndices
\partapendices

% ----------------------------------------------------------
\chapter{Quisque libero justo}
% ----------------------------------------------------------

\lipsum[50]

% ----------------------------------------------------------
\chapter{Nullam elementum urna vel imperdiet sodales elit ipsum pharetra ligula
ac pretium ante justo a nulla curabitur tristique arcu eu metus}
% ----------------------------------------------------------
\lipsum[55-57]

\end{apendicesenv}
% ---


% ----------------------------------------------------------
% Anexos
% ----------------------------------------------------------

% ---
% Inicia os anexos
% ---
\begin{anexosenv}

% Imprime uma página indicando o início dos anexos
\partanexos

% ---
\chapter{Morbi ultrices rutrum lorem.}
% ---
\lipsum[30]

% ---
\chapter{Cras non urna sed feugiat cum sociis natoque penatibus et magnis dis
parturient montes nascetur ridiculus mus}
% ---

\lipsum[31]

% ---
\chapter{Fusce facilisis lacinia dui}
% ---

\lipsum[32]

\end{anexosenv}

%---------------------------------------------------------------------
% INDICE REMISSIVO
%---------------------------------------------------------------------
\phantompart
\printindex
%---------------------------------------------------------------------

\end{document}
